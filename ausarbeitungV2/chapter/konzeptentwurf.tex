\chapter{Konzeptentwurf}
\label{cha:Konzeptentwurf}

Um ein passendes Konzept für die Umsetzung des Programms zu finden, wurden im Voraus einige Überlegungen, vor allem zur Steuerung der zeitlichen Abfolge angestellt.
Mit dem Konzept des Threadings kann sowohl die zeitliche Abfolge, als auch der Aufruf der einzelnen Funktionen in den Tasks effizient gesteuert werden. 

Die Abbildung  soll Threading bildhaft darstellen\footnote[1]{https://www.baeldung.com/wp-content/uploads/sites/4/2020/07/multithreading.png}	.

\begin{figure}[hbt]				% here, bottom, top
	\centering						% Zentrierung
	\includegraphics[width=0.6\linewidth]{images/multithreading}	
	\label{fig:multithreading}
	\caption{Threading}
	{\raggedright
	 }
	
\end{figure}

Der \glqq Main Thread\grqq \space startet mehrere untergeordnete Threads. Alle Threads greifen auf die gleichen Ressourcen und globalen Variablen zu. 

Am Anfang des Programms wird definiert, wie viele Tasks (bzw. Threads) es geben soll. Die genannte Anzahl Threads wird draufhin gestartet. Jeder Thread bekommt eine eigene Verkettete Liste, in der die einzelnen ausführbaren Funktionen mit einer Priorität gespeichert werden. Wird eine Funktion zur Liste hinzugefügt, sortiert ein Algorithmus die Funktionen nach Priorität. 
Die Threads durchlaufen alle die gleiche Funktion, in der durch die Listen iteriert wird und die Funktionen abgearbeitet werden. Anschließend wird der Thread so lange schlafen gelegt, bis der nächste reguläre Zyklus ansteht. 
Durch die Verwendung von Mutex werden Race-Conditions verhindert.
Die Threads laufen jeweils in einer Endlosschleife, damit das Programm erste terminiert wird, wenn es der User abbricht. 
Wird das Programm von außen terminiert, wird der von den Listen allozierte Speicher wieder freigegeben. 