\chapter{Tests}
\label{cha:tests}

Um das vorliegende Programm zu testen wird dieses als erstes Kompiliert, um sicherzugehen, dass keine Kompilerwahrnungen oder Errors auftreten. 
Ein weiterer Testfall ist das Einfügen eines \texttt{sleep} in eine Funktion, das eine aufgerufene Funktion so lange blockiert, dass die eigentliche Zykluszeit abgelaufen ist. Das Programm reagiert wie erwartet und gibt ein Fehler auf der Konsole aus, dass die Funktion zu lange für die Ausführung benötigt hat und wird erst wieder neu gestartet, wenn der nächste reguläre Zyklus ansteht. 
Außerdem wird die Erweiterbarkeit der Tasks und Funktionen überprüft. 
Die Selbstregistrierung von Funktionen wird getestet und funktioniert ordnungsgemäß. Beim Starten des Programms wird auf der Konsole eine Tabelle ausgegeben, welche genau zeigt, welche Funktionen in welchem Task konfiguriert sind.  