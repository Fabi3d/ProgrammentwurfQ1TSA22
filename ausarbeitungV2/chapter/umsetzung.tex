\chapter{Umsetzung}
\label{cha:Umsetzung}
\section{pthread}
Um erste Versuche mit Threading zu starten, wird ein Programm erstellt, das zwei Threads startet. 
Dabei wurde die Bibliothek 'pthread.h' verwendet. 

\texttt{pthread} steht für ' POSIX Threads' und ist eine API (Application Programming Interface), die es ermöglicht, in C und C++ Threads zu erstellen und zu verwalten. POSIX steht dabei für "Portable Operating System Interface", was bedeutet, dass diese API auf vielen verschiedenen Betriebssystemen verfügbar ist und somit portabel ist. Pthreads werden unter UNIX-basierten Betriebssystemen wie MacOS oder Linux unterstützt. 

Die \texttt{pthread}-API stellt eine Reihe von Funktionen bereit, mit denen Threads erstellt, gesteuert und synchronisiert werden können. Einige wichtige Funktionen der \texttt{pthread}-API sind:

\begin{itemize}
\item \verb|pthread_create|: Diese Funktion erstellt einen neuen Thread und führt eine bestimmte Funktion aus, die dem Thread als Argument übergeben wird.
\item \verb|pthread_join|: Diese Funktion blockiert den aufrufenden Thread, bis der angegebene Thread beendet wurde.
\end{itemize}

Die \texttt{pthread}-API ist sehr leistungsfähig und wird häufig in Anwendungen eingesetzt, die viele gleichzeitig ausgeführte Threads erfordern, wie beispielsweise Serveranwendungen oder Multimedia-Anwendungen\footnote[2]{https://www.cs.cmu.edu/afs/cs/academic/class/15492-f07/www/pthreads.html}. 

Als Threadfunktion wird eine Funktion erstellt, die eine Liste der vorgegebenen Struktur erstellt und dann eine Endlosschleife bis zur Terminierung von außen durchläuft. In dieser Endlosschleife wird zuerst überprüft, ob in der Liste Funktionen zur Ausführung hinterlegt sind. Sind alle Funktionen abgearbeitet wird der Thread so lange schlafen gelegt, bis der nächste periodische Zyklus der angegebenen Zykluszeit ansteht, um Systemressourcen zu sparen und den Jitter zu minimieren. 
Die Knoten der jeweiligen Listen haben folgende Parameter: 
\begin{itemize}
\item Einen Funktionspointer
\item Eine Priorität der Funktion
\item Einen Pointer auf den nächsten Knoten
\end{itemize}
Wird eine der vier Beispielfunktionen zu einer Liste hinzugefügt, wird die Liste anschließend unverzüglich neu sortiert. Hier kommt ein Bubblesort Algorithmus zum Einsatz.  

Funktionen können auf zwei unterschiedlichen Wegen zu einer Liste hinzugefügt werden. 
Es kann zum einen eine "reguläre" Registrierung der Funktion durchgeführt werden. Hierzu wird die Funktion \texttt{add_function(&task[], function, priority);} . Die Priorität legt die Reihenfolge der Ausführung fest. Die höchste Zahl stellt hierbei auch die höchste Priorität um. 

Außerdem kann eine Selbstregistrierung innerhalb der Funktion stattfinden. Hierfür wird die Funktion \texttt{selfRegister(function, priority, thread);}. Wichtig ist, dass der erste Thread der Null entspricht. 