\chapter*{Kurzfassung} %*-Variante sorgt dafür, das Abstract nicht im Inhaltsverzeichnis auftaucht

Im Rahmen des Programmentwurfs der Vorlesung Programmieren soll ein rudimentäres Zeitscheibensystem entworfen werden.
Hierbei sind einige Anforderungen zu erfüllen, wie zum Beispiel eine sehr große Variabilität des Programms, sodass mit möglichst wenig Aufwand neue Tasks angelegt, oder zu anderen Zykluszeiten umgesetzt werden können. 

Das vorliegende Programm basiert hierbei auf dem Konzept des Threadings um die zeitliche Abfolge zu organisieren. 
In den einzelnen Tasks können mathematische Funktionen mit einer bestimmten Priorität, also Vorgabe der auszuführenden Reihenfolge, konfiguriert werden.
Außerdem können sich die einzelnen Funktionen selbst für die Ausführung in einem Task mit einer gewissen Priorität registrieren.
Braucht eine Funktion länger für die Ausführung die die angegebene Zykluszeit, wird der Task erst dann neu gestartet, wenn er regulär wieder an der Reihe ist.

Abschließend wurde der Worst-Case Jitter der einzelnen Tasks berechnet.

\cleardoublepage
